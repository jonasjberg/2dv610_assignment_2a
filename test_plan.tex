% ______________________________________________________________________________
%
%   2DV610 Software Testing -- Assignment 2 Task 1
%
%   Author:  Jonas Sjöberg
%            Linnaeus University
%            js224eh@student.lnu.se
%            github.com/jonasjberg
%            www.jonasjberg.com
%
%  License:  Creative Commons Attribution 4.0 International (CC BY 4.0)
%            <http://creativecommons.org/licenses/by/4.0/legalcode>
%            See LICENSE.md for additional licensing information.
% ______________________________________________________________________________


\section{Test Plan}
%
% TODO: Write a test-plan for the application for current iteration.
%


% Test Plan -- Should describe and outline the current iteration’s test effort(Who does what when).
%
% Does the test-plan describe..
%
% * what requirements that should be tested?
% * what requirements that should NOT be tested?
% * how testing should be done? at what test-level?
% * who should do different activities? (responsibilities)


% ______________________________________________________________________________
\subsection{Requirements}
The total set of requirements is those specified in the requirements document
\cite{2dv610:assignment2-requirements} as well as additional requirements
inferred from the scenario description text
\cite{2dv610:assignment2-instructions}.

Total set of stated requirements;
\begin{enumerate}
  \item The web server should be responsive under high load.
  \item The web server must follow minimum requirements for HTTP 1.1
  \item The web server must work on Linux, Mac, Windows*.
  \item The web server must be easy to deploy.
  \item The web server must be easy to integrate into various systems.
  \item The web server must not require complex configuration.
  \item The web server must be absolutely secure.
\end{enumerate}


Not all of the requirements are testable or are likely to yield any actual
value, even though they are testable; automated tools for determining how
source code is licensed \emph{do} exist, but this is obviously not a priority
in this case.

The following requirements should be transformed into actual test cases;
\begin{enumerate}
  \item The web server should be responsive under high load.
        % Stress tests, Smoke tests.
  \item The web server must follow minimum requirements for HTTP 1.1
        % Unit tests, Integration/System tests.
  \item The web server must work on Linux, Mac, Windows*.
        % OS-specific Integration/System tests.
  \item The web server must be absolutely secure.
        % Fuzzing, Property-based tests, Penetration tests, Security Auditing.
\end{enumerate}



% ______________________________________________________________________________
\subsection{Mapping Requirements to Testing}
\subsubsection{The web server should be responsive under high load}
Testing of this requirement should be prioritized. Load testing clearly
correlates to robustness which in turn is directly linked to security.

Stress/smoke testing is a cost-efficient method that will provide a great deal
of insight into the servers operational characteristics.


\subsubsection{The web server must follow minimum requirements for HTTP 1.1}
Testing this requirement as a static quality could be done rather laboriously
by writing test cases for the HTTP specification \cite{rfc2616}. This is not
practically feasible given the allocated resources.

It would probably be more suitable to verify the requirements by testing
``dynamic qualities'' through integration/system testing, with test
cases that mimic client behaviour using arbitrary HTTP requests, etc.
The number of test cases would be extended as far as possible; with a
theoretically ``maximally extended'' test suite being practically tantamount to
fine-grained testing of the entire HTTP protocol specification.


\subsubsection{The web server must work on Linux, Mac, Windows*}
This requirement would be best implemented with OS-specific integration/system
tests. Preferably based on automated testing in virtual machines.

This places constraints on the tools used for testing. The given application
and test code is both written in Java, which should run pretty much the same on
all the target platforms.

The target platform, being an embedded system, might have very limited
available resources in terms of storage and computational overhead --- possibly
to the point of thwarting delivering the test code and its dependencies
(\texttt{Mockito}) bundled with the server. This must be taken into
consideration in order to satisfy testing goals of all stakeholders.

There are innumerable variants of embedded systems and we should inquire about
the platform specifications or a range of probable characteristics in order to
better plan for any in-system testing.


\subsubsection{The web server must be absolutely secure}
Reducing security vulnerabilities and eliminating bugs are closely related; it
is argued\cite{torvalds-security-morons} that ``security problems are just
bugs'' \cite{torvalds-security-bugs} --- eliminating bugs will by definition
reduce risks related to the security requirements, all testing will be related
to this requirement.

All previously mentioned tests are relevant here and addition to these, some
specific testing could be employed. Examples of this is fuzzing and
property-based tests, similar to functionality provided by \texttt{Mockito}.

Even more security-centric testing techniques like penetration testing and
similar security audits would be required to verify the ``absolute security''
stated in the requirements. However, this is probably outside the scope of this
assignment, given the allocated resources.

As in the previous requirement, the characteristics of the target architecture
should inform planning of in-system/integration tests.  The security
requirement in particular, places extra emphasis on testing the software,
running on the actual target system.

Writing the server code in (ostensibly platform-independent) Java does not
change the fact that \emph{the hardware is the platform}\cite{mike-acton-three-big-lies}.


\subsection{Current Test Plan}
These are the prioritized testing methods with corresponding requirement.

\begin{enumerate}
  \item Stress Tests --- be responsive under high load
  \item Unit Tests --- follow minimum requirements for HTTP 1.1
  \item Integration Tests --- must work on Linux, Mac and Windows
\end{enumerate}

It is assumed that the Java run-time and libraries is a totally opaque,
``non-leaky'' abstraction of the underlying operating system and hardware.
I.E., the computer hardware and operating system is considered irrelevant for
the time being.
All testing and development is done on a system running Linux
\texttt{4.4.0-104-generic x86_64} and Java SDK \texttt{OpenJDK 1.8.0_151}.


\subsection{Activities}
All responsibilities and tasks are assigned to the only available developer.
