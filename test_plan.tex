% ______________________________________________________________________________
%
%   2DV610 Software Testing -- Assignment 2 Task 1
%
%   Author:  Jonas Sjöberg
%            Linnaeus University
%            js224eh@student.lnu.se
%            github.com/jonasjberg
%            www.jonasjberg.com
%
%  License:  Creative Commons Attribution 4.0 International (CC BY 4.0)
%            <http://creativecommons.org/licenses/by/4.0/legalcode>
%            See LICENSE.md for additional licensing information.
% ______________________________________________________________________________


\section{Test Plan}
%
% TODO: Write a test-plan for the application for current iteration.
%


% Test Plan -- Should describe and outline the current iteration’s test effort(Who does what when).
%
% Does the test-plan describe..
%
% * what requirements that should be tested?
% * what requirements that should NOT be tested?
% * how testing should be done? at what test-level?
% * who should do different activities? (responsibilities)

\subsection{Requirements}
The total set of requirements is those specified in the requirements document
\cite{2dv610:assignment2-requirements} as well as additional requirements
inferred from the scenario description text
\cite{2dv610:assignment2-instructions}.

Total set of stated requirements;
\begin{enumerate}
  \item The web server should be responsive under high load.
  \item The web server must follow minimum requirements for HTTP 1.1
  \item The web server must work on Linux, Mac, Windows*.
  \item The web server must be easy to deploy.
  \item The web server must be easy to integrate into various systems.
  \item The web server must not require complex configuration.
  \item The web server must be absolutely secure.
\end{enumerate}


Not all of the requirements are testable or are likely to yield any actual
value, even though they are testable; automated tools for determining how
source code is licensed \emph{do} exist, but this is obviously not a priority
in this case.

The following requirements should be transformed into actual test cases;
\begin{enumerate}
  \item The web server should be responsive under high load. --- Stress tests, Smoke tests.
  \item The web server must follow minimum requirements for HTTP 1.1 --- Unit tests, Integration/System tests.
  \item The web server must work on Linux, Mac, Windows*. --- OS-specific Integration/System tests.
  \item The web server must be absolutely secure. --- Fuzzing, Property-based tests, Penetration tests, Security Auditing.
\end{enumerate}


\subsection{Mapping Requirements to Testing}
\subsubsection{The web server should be responsive under high load}
--- Stress tests, Smoke tests.

\subsubsection{The web server must follow minimum requirements for HTTP 1.1}
--- Unit tests, Integration/System tests.

\subsubsection{The web server must work on Linux, Mac, Windows*}
--- OS-specific Integration/System tests.

\subsubsection{The web server must be absolutely secure}
--- Fuzzing, Property-based tests, Penetration tests, Security Auditing.

