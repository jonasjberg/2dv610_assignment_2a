% ______________________________________________________________________________
%
%   2DV610 Software Testing -- Assignment 2 Task 1
%
%   Author:  Jonas Sjöberg
%            Linnaeus University
%            js224eh@student.lnu.se
%            github.com/jonasjberg
%            www.jonasjberg.com
%
%  License:  Creative Commons Attribution 4.0 International (CC BY 4.0)
%            <http://creativecommons.org/licenses/by/4.0/legalcode>
%            See LICENSE.md for additional licensing information.
% ______________________________________________________________________________


\section{Test-Strategy}
%
% TODO: Write a test-strategy for the application.
%
%       Does the strategy describe..
%
%       * stakeholders and their testing goals?
%       * the available resources?
%       * how the stakeholders testing goals are to be fulfilled by the plan?

This testing strategy is based on the scenario given in the assignment instructions.
The scenario \cite{2dv610:assignment2-instructions} is stated as follows:

\begin{quote}
  The small Software Development Company (SDC) has found a possible market niche
  for giving out a simple to deploy web-server. SDC aims to redistribute this
  server on a wide range of Internet Of Things (IOT) to present information
  from sensors etc. SDC wants an easy to deploy java-web-server that can be
  deployed on many different devices and therefore that can attract attention
  of a wide range of IOT developers. IOT-developers want minimal
  configuration as well as easy integration and adaptation of the web-server.
  End-customers want easy access and absolute security. The SDC management
  has found an open source abandonware software called “My web server”. It is
  your job as SDC employee(s) to evaluate the current state of “My web
  server”. SDC needs to know if the abandoned software fulfills the
  requirements as stated in the requirement document. Your budget is one
  man-week times the number of students. SDC management want a strategy,
  plan, test-cases as well as a test-report by the 23 December.
\end{quote}

\subsection{Stakeholders}
The following stakeholders can be identified in the scenario:

\begin{enumerate}
    \item SDC management
    \item SDC developers
    \item IOT developers
    \item End-customers
\end{enumerate}

Considering that the SDC management and SDC developers will have different
primary testing requirements, I've chosen to list them separately.


\subsection{Testing Goals}
\subsubsection{SDC management}
The goals and requirements of the SDC management will likely partially overlap
that of the SDC developers.

Management will probably appreciate if testing results and feedback is
presented in a less \emph{technically dense} format. A possibility is providing
test reports through a web interface, as an alternative to having the
developers be the ``gatekeepers'' to test reports and related information.

Technical specifics will not provide much insight to non-developers. Testing
should rather indicate whether the project is on schedule, which parts of the
project works and which does not work; a very high-level overview.


\subsubsection{SDC developers}
Developers will benefit from unit testing during refactoring and evaluation of
the legacy system. Unit tests can be used to sanity-check application
boundaries/interfaces and also provide insights when getting acquainted with
the code base.

A great number of testing techniques would be useful for developers given this
specific task.  But the allocated time and budget means that some concessions
must be made.

Unit tests are pretty much a given, but additional system tests or integration
tests are also required.
The entire system should exercised in order to test the interactions between
the units, covered by unit tests.

Seeing as how the software is intended to be used in ``Internet of Things''
applications, testing for security vulnerabilities will be very important.
These kind of embedded systems are widely known to be insecure for a variety of
reasons. \cite{iot-malwarebytes}
An \emph{absolutely secure} embedded system with a persistent internet
connection (``IoT'' device) is practically an oxymoron. It is widely known that
these kind of devices are easy targets for automated attacks due to the
unavoidable practical aspects of the medium --- mass-produced single-purpose
computers that most likely will not be properly maintained and updated.
Mass-production have often led to re-used or low entropy encryption keys
\cite{iot-crypto-key-reuse} and most people do not take the time to keep their
``IoT'' devices updated \cite{iot-attack-waiting-to-happen}.

With high security among the requirements for this type of product, basic
security auditing should be part of the testing.
In addition to unit tests and integration/system tests, some kind of
``fuzzing'' or property-based testing \cite{hypothesis-testing} could be used
to identify security vulnerabilities, at least some low-hanging fruit.

In addition to this, manual or automated penetration testing of the web server
would help to reveal obvious security flaws.
To meet the requirement of \emph{absolute security}, security auditing should
be done on the actual target platform and some of the software testing should
also be performed with the software running on the actual target
device\cite{6006307}.

Assuming that the Java virtual machine is a perfect abstraction of the
underlying hardware and ignoring any differences between JVM implementations,
OS schedulers and a multitude of other factors; the SDC developers could settle
with doing development and testing on a desktop PC.


\subsubsection{IOT developers}
% TODO: ..


\subsubsection{End-customers}
% TODO: ..
